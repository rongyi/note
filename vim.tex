\section{概述}
\subsection{Vim是什么}
文本编辑器
\subsection{为什么要用Vim?}
对于Hackers来讲一般这个问题他们也许会笑而不语,我不是,所以我回答:提高效率,对于文本编辑者,Vim或者Emacs你选一个,学会了,受用终身。
\subsection{本质}
本质就是将在编辑时需要的一些操作尽可能的以$O(1)$的复杂度实现一些常用功能并封装成接口提供给你,完完全全用你的双手而不再使用鼠标。
\subsection{再罗嗦一些抽象}
从一个字母\rightarrow 单词\rightarrow 句子\rightarrow 段落\rightarrow 篇章,这些是文本组成的基本对象,Vim所有的命令都是操纵上述的5个对象。中文没有字母,直接从字开始。
\subsection{Vim的模式}
怎样区分你按下的一个键是要输入的内容还是告诉Vim这是一个操纵命令?简单讲,Vim引入了模式之分,在一个模式下,Vim把键盘输入解析成命令,另外一个模式下Vim把键盘输入当成文档的追加。\footnote{顺便唠叨一句,Emacs用Ctrl和Alt键组合来区分命令和文本},之前的命令模式官方名称为Normal模式。在任意时刻用$ESC$直达Normal模式,下文所述都是在Normal下的操作。编辑模式官方名称为Insert模式,在Normal模式下按$i$进入Insert模式。
\section{开工}
编辑涉及到的操作有增加/删除/,而其中又涉及到一个问题就是增加/删除的位置,而编辑位置不一定是光标当前所在的位置。所以需要有光标的移动功能,移动的步子有大有小,大了在Vim里也不会扯着蛋。下文就从移动开始讲起。
\subsection{移动}
一般都将$hjkl$放一起讲vim,这篇文章不一般,所以我不将$hjkl$放一起讲。\footnote{言下之意就是这篇文章也可能就是个水贴}步伐从小到大。
\begin{enumerate}
	\item 首先是字符间的移动,由两个:左移/右移,分别$h$,$l$操纵。注意:Vim\textbf{区分}大小写。
	\item 步子再大一点,单词间的移动:$w$跳到下一个单词首字符。$e$跳到下一个单词尾字符,$b$向前跳
	\item 前面讲过可编辑的基本对象,考虑一下Vim窗口对文本的显示,此时又会引入一些可编辑对象,比如行,比如当前显示内容,所以有些可编辑对象是由View产生的。所以下一个长一点的步伐可能不是句子,而是行,因为英语长句占好几行都是常有的事情。下面就不罗嗦。直接上表吧。
\end{enumerate}
\begin{table}
	\centering
	\begin{tabular}{l|l}
	\beginrule
	向下一行 & $j$ \\
	向上一行 & $k$ \\
	上一句 & $($ \\
	下一句 & $)$ \\
	上一段 & ${$ \\
	下一段 & $}$ \\
	当前显示窗口顶部 & $H$\\
	当前显示窗口底部 & $L$\\
	当前显示窗口中间 & $M$\\
	向下翻页 & $<c-f>$ \\
	向上翻页 & $<c-b>$\\
	文档顶部 & $gg$\\
	文档底部 & $G$\\
	换编辑文件& $:e another_file.txt$ \\
	

	\endrule
	\end{tabular}
	\caption{Vim“步伐”由小到大的快速命令}
	\label{tab:vim_step}
\end{table}


此处我没有向你讲述小键盘和PgUp/PgDn都可以实现字符/行移动与翻页,但我强烈不建议去用这些键去操纵Vim,我看过很多的Vimer/Emacser用这些键在操控,我只是觉得他们没有领略到这两个编辑器的精髓:可以让双手放在主键盘区去做任何编辑的工作。\footnote{对Emacs来讲或许要把编辑两字去掉} 同样,我也希望你定位一行的时候不要按住$j$/$k$不放让光标疯狂,尽量用$<c-f>/<c-b>$配合$H$/$M$/$L$
