\documentclass[a4paper,dvipdfm]{article}
%------------------------------------------------------------------------------------------------------
%\usepackage[symbol]{footmisc}这个宏包的作用是更改脚注的一些排版方式选项有: perpage stable side multiple para symbol ragged marginal flushmargin hang
%\usepackage{abstract}%这个红包的作用是双栏排版时,让摘要不这样,使用onecolabstract环境即可
\usepackage{booktabs}%for \toprule \midrule \bottomrule
\usepackage{fix2col}%修补在双栏排版时无法保证浮动体的排印顺序
\usepackage{fontspec,xltxtra,xunicode}    % 这里使用 xetex 相应的宏包
\usepackage{indentfirst} 					%缩进章节或者section里的第一段
\usepackage{url}
\usepackage{natbib} %cite URL
\usepackage{graphicx}
\usepackage{epsfig}
\usepackage[left=1.8cm,right=1.8cm,top=2cm,bottom=1.5cm]{geometry}
%\usepackage[left=0.1in, right=0.1in,  top=0.1in, bottom=0.1in, papersize={91.44mm, 121.92mm}]{geometry} %this is for kindle ebook
\usepackage{zhfont}                       % 这里调用 zhfont.sty 
%\usepackage[left=1.8cm,right=1.8cm,top=2cm,bottom=1.5cm]{geometry}
%\setzhmainfont{Microsoft JhengHei}
\setmainfont{Georgia}
\zhspacing

%以下是把论文中的一些参数中文化,比如摘要、参考文献等
\XeTeXlinebreaklocale “zh”
\XeTeXlinebreakskip = 0pt plus 1pt
\renewcommand\arraystretch{1.5}
\renewcommand{\contentsname}{目录}
\renewcommand{\listfigurename}{插图目录}
\renewcommand{\listtablename}{表格目录}
\renewcommand{\refname}{参考文献}
\renewcommand{\abstractname}{摘要}


\usepackage{tipa}
\usepackage{textcomp}





\pagestyle{empty}

\begin{document}
\title{Chromium源码分析}
\author{荣怡}
\maketitle

\section{源码获取与编译}

\subsection{获取源码}

文献\cite{url:getcode}中列出了获取源码的地址\cite{url:downloadpage}步骤,具体做法可参照以下步骤:
\vspace{0.4cm}
\hrule
\begin{verbatim}
wget -c http://chromium-browser-source.commondatastorage.googleapis.com/chromium.r129764.tgz -O Chromiu.tgz
tar xvf Chromium.tgz
svn co http://src.chromium.org/svn/trunk/tools/depot_tools #下载更新工具
export PATH=$PATH:`pwd`/depot_tools
cd ./home/chrome-svn/tarball/chromium/src/ #这个目录是Chromium.tgz解压出来的层级目录
                                           #源码位于src目录下
gclient sync --force #更新chromium的代码,注意这个步骤中如果有提示某某文件夹已经不再被使用
                     #请删除,请删除对应的文件夹否则会造成编译错误。
#获取源码工作结束
\end{verbatim}
\hrule{}
\vspace{0.4cm}

\subsection{安装依赖}

源码获取后,需要安装依赖软件,可参考文献\cite{url:installdeps}。此外在link阶段会消耗大量内存,需要更换ld为ld.gold。在Ubuntu上的具体安装依赖及更换ld方法如下:
\vspace{0.4cm}
\hrule{}
\begin{verbatim}
sudo apt-get install bison fakeroot flex g++ g++-multilib gperf \
  libapache2-mod-php5 libasound2-dev libbz2-dev libcairo2-dev \
  libdbus-glib-1-dev libgconf2-dev libgl1-mesa-dev libglu1-mesa-dev \
  libglib2.0-dev libgtk2.0-dev libjpeg62-dev libnspr4-dev libnss3-dev \
  libpam0g-dev libsqlite3-dev libxslt1-dev libxss-dev \
  mesa-common-dev msttcorefonts patch perl pkg-config python \
  python2.5-dev rpm subversion libcupsys2-dev libgnome-keyring-dev \
  libcurl4-gnutls-dev libelf-dev libc6-i386 lib32stdc++6

sudo apt-get install apache2 wdiff lighttpd php5-cgi sun-java6-fonts \
   msttcorefonts ttf-dejavu-core ttf-kochi-gothic ttf-kochi-mincho

sudo apt-get install ttf-indic-fonts
sudo cp /usr/bin/ld{,_bk}
sudo cp /usr/bin/ld.gold /usr/bin/ld #you can also use ln
\end{verbatim}
\vspace{0.4cm}
\hrule{}


\subsection{编译}
编译步骤:先用google的gyp工具生成Makefile,然后make编译
\vspace{0.4cm}
\hrulefill{}
\begin{verbatim}
./build/gyp_chromium
make chrome #默认编译debug版本
make chrome BUILDTYPE=Release #如果想编译Release版本加此编译选项
make test_shell #编译webkit测试程序,与chrome的区别是render和broweser在同一个进程,方便
                #调试
\end{verbatim}
\hrulefill{}
\vspace{0.4cm}
\bibliographystyle{plainnat}
\bibliography{chromium}
\end{document}

